\documentclass{article}
\usepackage{amsmath}
\usepackage{titling}
\begin{document}
\title{le avventure di lorenzo nella terra del DL}
\date{\today}
\author{kotatsu}
\maketitle
\section{CIAO A TUTTI}
\begin{itemize}
	\item per prima cosa cerchiamo di fare una roba decente, creiamo un dataloader e aumentiamo il dataset
	\item quindi, a cosa serve il dataloader? in realtà non a molto, serve a dividere automaticamente in batches e shufflare durante le epoche
	\item ora come ora il punto sta nel creare sti dataset aumentati
	\item ora brutta testa di cazzo devi avere i file già divisi
\end{itemize}
\end{document}
